For this assignment I have chosen to focus on topic A (Health Care). We see that today the Internet of Things (IoT) appears to be the next big thing, and it is somewhat already surrounding us with more and more smart devices in our daily lives. The IoT term is widely used and originated in 1999 when Kevin Ashton used the term for the first time [?] when he described an idea of a combination of the Radio Frequency ID (RFID) and the Internet. The IoT is a broad term and is an umbrella keyword that covers various aspects related to the extension of the internet. As the name IoT implies, it is all about interconnecting anything, anytime and anyhere. It represents numerous new opportunities and will create an ever-increasing network of things and devices. In a forecast by Gartner Inc. they state that 4.9 billion connected things will be in use in 2015, which is more than 30 percent up from 2014, and that it will reach 25 billion by 2020 [?]. One of the domains in the IoT is the healthcare domain, where the application of data collection is an important aspect [2].
With the IoT, health care can be revolutionized. As a big part of IoT, new smart medical devices are continuing to show up. This include a device that reminds you to take your medication dose, checking blood pressure, taking a walk, or doing your cardio at a scheduled time. These devices have a huge potential. They train you about a procedure, provides health advises and other informations. If we look at fx. home care as it is today, we often have a public or private company, that visits patients at their home, on daily checks to see how the patients are doing. By using health sensors a patient could record data about their blood pressure, heart rate, weight, glucose levels and other relevant data. These data can then be send directly to a caretaker from the hospital, for further analysis about what the physically status of the patient is.
Another application of the IoT is fitness and sports tracking, where fitness tracking devices are used to gain information about an athletes performance. The fitness and sports tracking also helps people to maintain a healthy lifestyle and thus it contributes to the general health in the society. The sports tracking is also used in professional sports such as running and cycling where the athletes receives information about their heart rate, cadence, etc.
Most of the data that is collected in these health care systems are sensitive and personal data, and therefore they needs to be handled with care. One single point of failure when handling the security of the sensitive data is the transportation of the data and the storing of the data. In this report I will focus on the transportation of the data in particular the transportation from a sensor to a head processing unit. There are several protocols used for this transportation such as Bluetooth, ZigBee and Ant+. I have chosen to look at the Ant+ protocol, which is a popular and emerging protocol especially in terms of personal health care and fitness and sports tracking.
2
1.1 Structure of the report
This report is structured such that in section 2 I will describe two scenarios that could be thought of in terms of an IoT system in the health care domain. For both scenarios I will analyse the possible threats and their consequences to the systems. For both scenarios I will only be looking at the communication between the sensors and the data collecting device. In section 3 I will look into the ANT+ protocol, and based on the threats described in section 2 I will look at how these threats are handled by the protocol. In section 5 I will look at the Danish personal data protection act and look into the the preparations and considerations that needs to be taken into account when implementing the scenarios described in section 2. In section 6 I will discuss the issues that has been addressed and then in section 7 I will wrap up the report.
2 Internet of Things scenarios
As stated in the introduction, the Internet of Things has brought a lot of new possibilites with it, and especially the increasing sensor devices brings possibil- ities in the health care domain. I will here propose two IoT scenarios, the first being a remote diagnostics of elder people, and the second being an IoT scenario in professional cycling. Both of the scenarios can use the wireless technology ANT+.
2.1 Remote diagnostics health care
The topic of health care is an important topic as the tendency in society is that people are getting older and older. The Ageing Report from the European Commission [5] states that the amount of elderly people is becoming bigger in the population. A forecast states that the percentage of the total population at the age of 65 will increase from 18% in 2013 to 28% in 2060 of the population, and the percentage of the population over 80 years of age will increase from 5% to 12%. With this trend, it is expected that we need to spend more resources and money on health care to help the elderly. Because of a trend like that, it is of high interest to utilize the advantages of the IoT in health care in terms of an increased health care for the elderly as well as a cheaper health care for the society.
A well-known scenario of utilizing the IoT in health care is to make use sen- sors to record data from patients. Measuring parameters such as weight, blood pressure, heart rate, activity (daily steps that has been walked, and cadence when cycling), gives an indication about elderly patients well being. Similarly other systems targeted at for example diabetics patients could be developed. This could be realized by adding other sensor devices for tracking the glucose level, and maybe the ability to remotely or automatically control the dozes of insulin a given patient would need.
Figure 1 illustrates a remote diagnostics system for elderly people. This system has sensors that collect data and then this data is sent to a remote
3
expert such as a doctor or a nurse. ANT+ is used for collecting the sensor data into the data collecting device, and the data could then be sent to the remote doctor via the Internet with a protocol such as the HTTP protocol.
Figure 1: A remote diagnostics health care system for elderly people.
In a remote diagnostics health care system like the one described, the data need to be collected from the sensors, into a data collecting device that can store the data. All of the stored data would then need to be sent remotely to a doctor for analysis, and maybe some friends and family that take care of the person would be interested in seeing this data too. Such a system will ease the public health care, since it will reduce the need for people checking up on the patients, plus it will produce a whole lot more data for the doctors to analyse.
Wireless Sensor Network The system I have described can be implemented as a Wireless Sensor Network. This means that all the sensor data is collected wireless in a personal network to a master node. Such a network would often have a reach around a few metres. In the scope of this report I will mainly focus on the data collecting from the sensors via the protocol ANT+.
2.1.1 Security aspects
With a system like the health care system, some potentially very sensitive per- sonal data will be handled. Therefore there is a need for high security. The potential types of threats to a system fall in the categories of confidentiality, integrity and availability [1]. In this section I will look at the consequences of harmful attacks to the system in the three categories, in the scope of trans- portation of the data with ANT+.
Confidentiality The confidentiality of a system is the system’s ability to protect the sensitive data from unauthorized access. In our health care system, we are handling personal data. Although not all the data is sensitive, the data could have some special value to some people. Not all people would be interested in weight data or heart rate data to be publicly known, especially if the patient has some issues related to the data. So basically we would not want people to be able to get their hands on the data. The typical way of enforcing the confidentiality of a system is by use of encryption in the communication medium.
4
 
Integrity The integrity of a system describes whether the data in the system is trustworthy, and that it has not been modified with in an unauthorized manner. Integrity of the data in the health care system is a very important security aspect. This is due to that the data collected from the sensors needs to be reliable, for a doctor to be able to analyse the data and come to a conclusion that is consistent with the patients state of health in real life.
A harmful attacker could want to do modification or fabrication attacks and generate data that would make the doctors worried for the patients. This could lead to the doctor sending somebody out to the patient to do a check on the patient, that he or she is allright. If we have a scenario with a lot of elder people living in senior housing, an attacker could make this kind of attack, an cause a lot of extra costs for the hospital by alarming them with misleading values all the time. The attacker could also modify the data in a way so the values looks correct, and causing the opposite effect, where a patient would not get the help that is needed.
Attacks on the integrity can invoke false alarms, which has to be avoided.
Availability The availability of the health care system is also an important aspect. Much like the modification or fabrication attacks, if the health care system is being attacked on the availability it will cause the data cannot be sent to the doctor. If the the length of such an attack is long it would mean that the data would not be recorded. This again would result in the hospital having to send someone to the patient to check if everything is okay, which is a waste of resources.
2.2 Professional cycling case
In professional cycling there is also a great need for collecting data about the riders. This data could similarly to the suggested remote diagnostics health care system be the heart rate, cadence, speed and power. Recently there has also been developed gear shifters that communicate wireless via the ANT+ protocol [4]. All of this data will be collected in a data collecting device, which is the cycling computer. One could also imagine that the wireless shifters would be intelligently controlled by the cycling computer, based on data that the computer has collected. The data collected from a professional cyclist might not be as sensible as the data in the health care system, but to a cyclist the data could for competitive or personal reasons need to be kept to himself and therefore the data to some extend the data needs to be protected.
2.2.1 Security aspects
It might not be in the interest for a professional top performing cyclist to share his physical data with his direct contenders during a race. This is due to the psychological game between two contenders, which would be revealed to some extend if they know their contenders physiological data. As it is today, most of the professional cyclists use the ANT+ protocol in their cycling computer from
5
e.g. Garmin or SRM [6] [7]. In the case of professional cycling I will also look at thesecurity in terms of the confidentiality, integrity and availability.
Confidentiality If a lot of cyclists ride together in a peleton and they all are getting their data transmitted wirelessly, it is important that a cyclist only receives his own data and that no one else is able to receive his data other than him and other persons granted access such as a the particular sports director. Therefore the confidentiality is an important aspect of the security in such systems, since the rider needs to trust the data he sees is the right data.
Integrity In order for the rider to be able to trust the data he sees on his cycling computer the integrity needs to be well protected. An opponent with bad intentions could manipulate his data, and maybe gain some advantage, if the cyclist is very dependent of his data. Of course this might not be of uttermost importance, since the rider always can listen to his body, but in a professional sport it is important that the athletes can trust their gear. If we think of the case of wireless controlled gears, it would be very important that the communication is well protected. If an opponent is about to do a breakaway, and he can control the cyclists gear shifting, then he would be able to gain huge advantages in the race.
Availability The availability is certainly an important aspect if we consider professional cycling. Some riders are very dependant on their data that they see on their cycling computer in order to know how much and for how long they can perform at a given pace.
3 The ANT+ protocol
ANT+ is the wireless technology that allows sensors and other monitoring de- vices to talk together. ANT+ defines the way data is represented and is respon- sible for the application layer of the OSI model. These definitions are known as profiles in ANT+. A profile typically describes a specific use case such as heart rate or blood pressure. The ANT+ protocol is an extension to the ANT protocol that operates in the 2.4 GHz spectrum like other well-known protocols such as Bluetooth and WiFi. The ANT protocol takes care of the Physical, Network and Transport layers in the OSI model as seen on figure 2. As with all wireless protocols, range is always a issue. The range is affected by a number of factors such as the antenna chosen and environmental obstacles. The range in ANT+ is typically 10-30 meters.
Channels The communication between nodes in ANT is done on channels that has a configuration. The configuration of an ANT channel describes among other things the channel type, the channel frequency, the channel ID, and the network key. These parameters together will make an ANT channel somewhat
6
 Figure 2: The OSI layers that is taken care of in ANT and ANT+.
unique. The channels are the ANT protocol’s most fundamental building block. A channel is the medium through which two nodes can communicate. Thus a channel needs a master node and a slave node. There is 125 unique radio frequencies to use for the channel frequency. Channels has the ability to be bidirectional, shared bidirectional or transmit/receive only. These properties allows for several topologies with the ANT protocol such as peer-to-peer, star, tree and mesh.
3.1 Data types
There are four data types supported by ANT: broadcast, acknowledged, burst and advanced burst data. Each data type is sent in 8 byte packets over the channel. ANT channel is not restricted to a single data type meaning that all data types can be sent on each type of channel, except unidirectional channels, which can only send broadcast data.
Broadcast With the broadcast data type the data is sent in one-way from one node to another. The receiving node will then not transmit any acknowledge- ment. This technique is suited for sensor applications such as in the scenarios described in section 2. The broadcast data type is also the most used by the ANT sensors.
Acknowledged The acknowledged data type send an acknowledging message that confirms receipt of data packets. The transmitting node is then informed of success or failure of the message. This technique is suited to control applications, such as the gear shifting in the professional cycling scenario, or for a dosing machine in a health care system. The acknowledged data type might also be used in a health care system, where it is important that the sensor knows that the data has been received.
7
Burst With the burst data type the messages are numbered sequentially. This technique is suited to data block transfer.
4 Security in ANT+
In this section I will look at the security of ANT+. This is done by looking at what ways and with what security mechanisms the ANT+ protocol enforces the confidentiality, integrity and availability of the data that is being sent. The ANT+ protocol offers encryption, but this is not used by default and is not applicable to all the channels.
Because of the short range that the ANT+ offers with the wireless trans- portation, possible attackers needs to physically be in the range of the sensors and the data collecting device that is communicating. This is one of the good properties of the personal area networks. However one could use equipment to pickup radio frequencies from a wider distance. When considering the range of wireless radio frequency waves this range could of course be eliminated by using wired communication. However since the wireless property is very convenient and increases the usability in applications such as the health care system aimed for elderly people or for fitness tracking equipment this is a sacrifice that has to be made.
4.1 Confidentiality
The confidentiality is an important aspect in a health care system. We want to be sure that we know who has access to the data, and that we can be sure from where we have gotten the data. If we look at the ANT+ protocol the confidentiality is ensured through the combination of the channel frequency and the channel ID as well as the network key, by separating the legitimate users from each other by use of unique channels.
Channel ID and frequency The combination of the channel frequency and channel ID generates a unique channel for a slave to communicate with its master, in our case a sensor communicating with the data processing unit. The use of a unique channel for communication, offer some level of confidentiality. This is due to the fact that a slave will only establish a channel to the master it wants to communicate with, hence no other master will be able to receive data without the sensor’s approval. This however only prevents nodes with good intentions to get their hands on data from slaves they have not established a connection with.
Network key The network key is an 8-byte number used for uniquely iden- tifying a network. According to the ANT specification the network key can provide a measure of security and access control. The idea is that only channels with identical valid network keys is allowed to communicate together. The net- work key is a static key that is defined by ANT+ alliance. The network key is
8
by default set to be public, which means that everybody is able to listen in on the communication in a network. ANT uses default keys when pairing devices so that they know if it is a private network, ANT+ network, ANT-FS (ANT File Sharing) network or a public network.
Encryption An important and essential way of protecting the confidentiality of a system is by applying encryption to the communication channels. For doing so ANT offers the single channel encryption, where a 128-bit AES-CTR encryption is used. This definitely is a good countermeasure against attacks on the confidentiality. However the encryption of channels is often not applied in the ANT protocol, but in case of the health care system it would be required, and mostly preferred in a professional cycling scenario.
4.2 Integrity attacks
The integrity is important according to the scenarios described. As mentioned in the confidentiality the ANT+ supports encryption with 128-bit AES. However when looking in the specification of ANT [9] it can be seen that there is no real mechanism for the key exchange of the encryption key. This means that when an encrypted channel is being initialized, the encryption key is shared in clear text, which of course is a big problem for the integrity of a system using ANT+. By not using a key exchange algorithm, and just sharing the symmetric encryption key in clear text, an attacker could eavesdrop the communication channel and get the encryption key. This leads to the attacker being able to perform the typically passive network attacks such as eaves dropping and traffic analysis, while the two nodes communicating would not know that anything is wrong, and having a false sense of security. Also active network attacks such as modification and fabrication could be done, which clearly would be a violation to the integrity.
Due to the limited range of the ANT protocol, it could be argued that the setup of the encryption could be done in a secure location where no attacker would be able to eaves drop. However, this is not always possible, if we look at the usability aspect for an elderly that needs to use the system, they properly would not go somewhere safe to setup the connection. Another aspect is that it is not easy to be sure that a place is secure. Also if a connection for some reason fails and needs to be setup again, for example for a rider in professional cycling who is in a race setting with fans, other riders and medias all around him it would not be possible to setup the connection safely. An attacker would also be able to do an availability attack as discussed later and then forcing the negotiation for a new connection to be set up.
The natural countermeasure to this insecure key exchange is to use a public key exchange. This means that each node would have its own public key and every message encrypted with this public key can only be decrypted with the nodes private key. A well-known public key echange algoritm is the Diffie Hell- man algoritm [1]. This algorithm however, does not by ensure authentication in any way. This means that when an encrypted channel is being set up, then a
9
given node cannot be sure of the actual identity of the node at the other end of the communication channel. This leads to vulnerability for man in the middle attacks. To solve the problem of reliable identities a public key infrastructure could be used. To ensure the identity of a given node it would then apply to a certificate authority for a digital certificate that proves the identity of the nodes public key. However considering ANT+ the devices are often limited devices (sensors), in terms of processing power and memory. They are often without a connection to the internet, which would make the use of a public key infrastructure difficult.
Since the ANT+ protocol is a propetary protocol it might perhaps be feasible for an ANT+ alliance to do signing of the nodes public keys. This however would need a lot of work since every node then would need to have a signed public key which would be very cumbersome.
4.3 Availability attacks
Since the ANT+ is a wireless communication protocol, an obvious attack on the availability comes from jamming the frequencies, that are used for communicat- ing. If an attacker is jamming the frequencies he is sending radio signals that disrupt communication between the nodes. Since he is introducing a lot of noise on the channel it becomes difficult for the nodes to distinguish the data from the noise. A counter measure to jamming is to use the technique of adaptive frequency hopping known from, inter alia, Bluetooth[3]. Frequency hopping is a mechanism, where the communication between two nodes will take place on dif- ferent frequencies. When changing the frequency in a random pattern, the new frequency will be categorized as either good or bad. By using this mechanism, the communication will avoid taking place on a frequency that is jammed.
However the ANT protocol does not provide a frequency hopping mecha- nism but instead ANT has a frequency agility mechanism. This mechanism allows a channel to change its operating frequency. The channel will monitor the channel’s performance and only change operating frequencies whenever the quality on the used frequency is bad. This approach is less expensive in terms of computing power. The mechanism is not a prevention mechanism like the frequency hopping, but rather a detection mechanism. In terms of preventing jamming, the frequency hopping mechanism is preferable due to the random pattern of frequency used and therefore it is hard for an atacker to know what frequency to jam. When using the ANT’s mechanism it is easy for an attacker to find the frequency to jam, and just as easy to find the new frequency that is being used. This mechanism was designed to improve coexisting of the ANT protocol together with other protocols used in the 2.4GHz spectrum, such as Bluetooth and Wi-FI.
Another well-known attack on the availability of a system is the flooding attack. The idea of an attacks such as the flooding attack is that by sending a flood of requests to a node and hereby using all the nodes capacity in terms of computing power or space. One of the well-known flooding attacks is the SYN flood that exploits the three-way handshake by sending a lot of SYN requests
10
to the target and using up all the targets resources.
It does not look like the ANT is vulnerable for flooding attacks due to its limited numbers of connection for a node, so a similar attack to the SYN flood attack, would not be possible. However this could be an interesting aspect to test the protocol for in a practicallly.
5 Personal data protecting act
If we look at the health care system we have a patient, a system for collecting data about the patient which is then sent remotely to a doctor. I have briefly mentioned earlier that this data is personal data. When determining whether data is personal, one of the important questions is whether the data can identify a living individual [8]. In case of the data in the health care scenario, the raw data might not have a direct link to the person himself. However when the data is collected and send to the doctor it definitely has some kind of link to the person, that could be the social security number. It could also be argued that if an attacker gets hands on the data, then the attacker would know who the person is since the patient will be at their home, and the attacker would then be able to derive the information from the settings.
When sensitive data is being handled there is some clear guidelines in the law. In Denmark the law is the personal data protecting act ”Persondataloven” [10]. The law states that there is the dataresponsible (dataansvarlige) and the data processor (databehandler).
Data responsible The data responsible is defined in ”Persondataloven” § 3, nr. 4. The definition is the that the data responsible is the person, private company, public company or institution that is responsible for deciding what personal data that must be processed
Data processor The data processor is defined in ”Persondataloven § 3, nr. 5. The definition is the person, private company, public company or institution that is handling data for a data responsible.
The law also states that the data responsible is responsible for implement- ing appropriate technical and organizational security measures to protect data against accidental or unlawful destruction or loss, alteration, the unauthorized disclosure, abuse or other processing in violation of the law.
6 Discussion
When we consider the use of technology to collect physiological data there is a lots of different solutions to do this. In terms of health care there has been
11
suggestions to use wireless sensor networks for gathering the data. When con- sidering such systems it is important that the system has a high reliability, since the systems are handling important data and in some cases vital. If we look at the remote diagnostic health care scenario we have the doctor who would be a data responsible. It is thus up to the doctor to be sure that the technical solution he is using is secure and not open to attacks. The doctor could also use a specific solution from a data processor, that would provide the IT-solution, but it is up to the doctor to be sure that the solution is secure.
There are several aspects of such systems that needs to be secure. In this report I have looked at the wireless communication from the sensor nodes to a data collecting unit as a single point of failure. Even though the raw data collected might not be the most important data for a patient, it is however important data in the sense of the information it contains. Another aspect is that if this data is modified, deleted or non available it can cause to wrong analyses and cause false alarms.
When we consider using the ANT+ protocol for such a system, there are sev- eral aspects of the protocol that is insecure as illustrated in section 4. Therefore the ANT+ protocol might not be the best candidate for health care systems, but it does however offer a high usability. If the ANT+ protocol should be considered a serious candidate for health care systems it needs however to get the security issues that is has fixed.
7 Conclusion
In this report I have looked at the emerging health care domain of the Internet of Things. Especially I have looked into two scenarios that both is taking the advantages of the technologies that emerges with the Internet of Things. The first scenario has been a elderly health care remote something where a wireless sensor network was used to gather data about the patient, which then could be sent to a remote doctor for analysis. The other scenario was the use of a wireless sensor network in professional cycling, where sensors has been used to gather data for the cyclist to use. Both of the scenario I have considered where to use the emerging ANT+ protocol, that targets the health care and fitness domain. In both scenarios I have looked at the threats to the security as well as the potential consequences of attacks. I have then analysed the ANT+ protocol for its ability to resist security threats in the light that it is a single point of failure in the scenarios I have created. Considering some of the security issues that the ANT+ protocol has, it might not be the best candidate for a health care solution, since such solutions deals with personal data. In this report I only looked at a small section of the upcoming health care solutions in the Internet of Things. It would be interesting to implement such a scenario and try to complete the suggested attacks on the protocol, when using the best security that the protocol offers.